\documentclass[a4paper,12pt]{article}
\usepackage{listing}
\begin{document}


    \title{14. Concurrent File Server}
    \author{Santhisenan A}
    \date{\today}
    \maketitle

\section{Aim}
Develop concurrent file server which will provide the file requested by client if it exists. If not server
sends appropriate message to the client. Server should also send its process ID (PID) to clients for
display along with file or the message.

\section {Concurrent File Server}
In computing, a file server (or fileserver) is a computer attached to a network that provides a location for shared disk access, 
i.e. shared storage of computer files (such as text, image, sound, video) that can be accessed by the workstations 
that are able to reach the computer that shares the access through a computer network. The term server highlights 
the role of the machine in the client–server scheme, where the clients are the workstations using the storage. 
It is common that a file server does not perform computational tasks, and does not run programs on behalf of its 
clients. It is designed primarily to enable the storage and retrieval of data while the computation is carried 
out by the workstations.

A server can be iterative, i.e. it iterates through each client and serves one request at a time.
Alternatively, a server can handle multiple clients at the same time in parallel, and this type of a server is 
called a concurrent server.

\section{Code}
\subsection{server.py}


\begin{verbatim}

import socket
import threading
import os

def recvFile(name, sock):
    filename = sock.recv(1024)
    if(os.path.isfile(filename)):
        sock.send("EXISTS" + str(os.path.getsize(filename)))
        userResponse = sock.recv(1024)
        if(userResponse[:2] == "OK"):
            with open(filename, "rb") as f:
                bytesToSend = f.read(1024)
                sock.send(bytesToSend)
                while(bytesToSend != ""):
                    bytesToSend = f.read(1024)
                    sock.send(bytesToSend)
    else:
        sock.send("ERR")
    sock.close()

HOST = '127.0.0.1'
PORT = 5000

s = socket.socket()
s.bind((HOST,PORT))

s.listen(10)

print "Server running..."
while True:
    conn, addr = s.accept()
    print "Client "+str(addr)+" connected"
    t = threading.Thread(target=recvFile, args=("retrFile", conn))
    t.start()

s.close()

\end{verbatim}

\subsection{client.py}
\begin{verbatim}
    import socket

HOST = "127.0.0.1"
PORT = 5000

s = socket.socket()
s.connect((HOST, PORT))

filename = raw_input("Enter the filename: ")
if(filename != 'q'):
    s.send(filename)
    data = s.recv(1024)
    if(data[:6] == "EXISTS"):
        filesize = long(data[6:])
        message = raw_input("File is present in the server. Do you want to download? (y/n) :")  
        if(message == 'y'):
            s.send("OK")
            f = open("recv_" + filename, "wb")
            data = s.recv(1024)
            totalReceived = len(data)
            f.write(data)
            while totalReceived < filesize:
                data = s.recv(1024)
                totalReceived += len(data)
                f.write(data)
            print "Download complete"
    else:
        print "File is not present in the server"
s.close()

\end{verbatim}
\section{Output}

\end{document}